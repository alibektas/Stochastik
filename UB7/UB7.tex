\documentclass{article}

\usepackage{amsmath}
\usepackage{amssymb}

\title{Übungsblatt 7}
\author{Richardo Adrian Budianto  583669 \and Junhyuk Ko  531806 \and Ali Bektas 588063 }



\begin{document}
	
	\maketitle

	\section*{Aufgabe 21}

	\subsection*{a)}
		Finden Sie $ X = max\{ T_1 , \dots , T_n \}$. 

		Um Max zu untersuchen fragt man jedesmal nach der Wkt dass alle Zufallsvariablen Werte weniger als t haben.

		\begin{align*}
			P(X \leq t) &= P( T_1 \leq t) \cdot P(T_2 \leq t) \dots P(T_n \leq t)\\
			&= (1-e^{-\lambda_1t})\cdot (1-e^{-\lambda_2t}) \dots \cdot (1-e^{-\lambda_nt})\\
			&= \prod_{i=1}^{n} e^{\lambda_i \cdot t}
		\end{align*}
	\subsection*{b)}
		Finden Sie $ Y = min\{ T_1 , \dots , T_n \}$. 

		\begin{align*}
			P(Y \geq t) &= P(T_1 \geq t) \cdot P(T_2 \geq t) \dots \cdot P(T_n \geq t) \\
			&= (1- P(T_1 \leq t)) \cdot (1-P(T_2 \leq t)) \dots \cdot (1-P(T_n \leq t)) \\
			&= \prod_{i=1}^{n} e^{-\lambda_i \cdot t}
		\end{align*}

		Beachte dass sich daraus ergibt dass Y wieder exponentialverteilt ist. Dementsprechend ist $E(Y)$

		\[
			Y = e^{-t(\sum_{i=1}^{n} \lambda_i)}  \rightarrow  E\textbf{Y} = \frac{1}{\sum_{i=1}^{n} \lambda_i}
		\]

	\subsection*{c)}
		\[
			E\textbf{X} = E[max\{T_1, \dots , T_n\}] = E\textbf{T}_1 = \frac{1}{\lambda} 
		\]

	\subsection*{d)}
		\begin{align*}
			P(X \leq 15) &= (1 - e^{-\frac{1}{20}})^5 \\
				&= (1-e^{-3}{4})^5 \\
				&\approx  0.0408\\
			P(Y \leq 15) &= (1- e^{-5 \cdot \frac{1}{20} \cdot 15})\\
			 	&= 1 - e^{-15}{4} 
			 	&\approx 0.9764 
		\end{align*}
			
	\section*{Aufgabe 22}

	\subsection*{a)}
		Da \textit{exp} überall differenziebar ist und $(e^x)' \neq 0 $ , ist ln auch differenzierbar.Somit ist g differenzierbar. Deshalb darf man den Transformationssatz anwenden.

		\begin{align*}
			h(y) &= \frac{f(g^{-1}(y))}{|g'(e^{-\lambda y})|} \\
			&= \frac{f(e^{-\lambda y})}{|-\frac{1}{\lambda \cdot e^{-\lambda y}|}}\\
		\end{align*}
		
		Bevor wir die endgültige Antwort liefern brauchen wir zu untersuchen wann $f(x) = 1$ gilt.
		\begin{align*}
			O &\leq e^{-\lambda \cdot y} < 1\\
			ln(0) &\leq -\lambda \cdot y < ln(1)\\
			-\infty &\leq -\lambda \cdot y < 0\\
			\infty &> y > 0\\
		\end{align*}

		Dann gilt:

		\begin{align*}
			h(y) &= 
			\begin{cases}
				\lambda \cdot e^{-\lambda \cdot y} &, \text{falls} y > 0,\\
				0 &, \text{sonst}.
			\end{cases} 
		\end{align*}

	\subsection*{b)}

	\begin{align*}
		g(U) &= \sqrt{-\frac{1}{\lambda}\cdot ln(U)}\\
		g^{-1}(y) &= e^{-\lambda \cdot y^2} \\
		g'(U) &= \frac{1}{2}\cdot\frac{1}{\sqrt{-\frac{1}{\lambda}\cdot ln(U)}} \cdot (-\frac{1}{\lambda U})\\
		h(y) &= \frac{f(e^{-\lambda y^2})}{|\frac{1}{2} \cdot \frac{1}{\sqrt{-\frac{1}{\lambda}}-\lambda y^2} \cdot (-\frac{1}{\lambda \cdot e^{-\lambda y^2}})|}\\
	\end{align*}

	Wann gilt $f(x) = 1 $?
	\begin{align*}
		0 &\leq e^{-\lambda y^2} < 1\\
		-\infty &< -\lambda y^2 < 0 \\
		\infty &> y^2 > 0\\
		y &> 0
	\end{align*}

	Dann gilt:
	\[
		h(y) = \begin{cases}
			&2\cdot y \cdot \lambda \cdot e^{-\lambda y^2} , fall y > 0\\
			& 0 , sonst. 
		\end{cases}
	\]
\end{document}