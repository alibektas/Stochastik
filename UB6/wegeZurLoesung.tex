\documentclass{article}

\usepackage{amstmath}
\usepackage{amsthm}

\newtheorem{theorem}{Theorem}
\newtheorem{definition}{Definition}[section]
\newtheorem{corollary}{Corollary}[theorem]
\newtheorem{lemma}[theorem]{Lemma}
\newtheorem{example}[theorem]{Example}

\begin{document}
	
	\section*{Aufgabe 18}
		$\bar{X} = \frac{1}{n} \cdot (X_1 +X_2 + \cdots + X_n)$. Bestimme Erwartungswert und Varianz von \bar{X} :


	\subsection*{Allgemeines Wissen}
		\begin{definition}{p-tes zentrales Moment}
			\[
				\textbf{E}X^p = 
					\begin{cases} 
						\int_{-\infty}^{infty} x^p \dot f(x) dx &, X stetig \\
						\sum_{i\in \mathbb{N}} x_i^p \dot p_i &, X diskret.
					\end{cases}
			\]
		
		2-tes Moment heißt \textbf{Varianz}.
		\end{definition}
		\begin{theorem}
			\[
				X_1 \text{und} X_2 \text{unabhängig} \rightarrow cov(X_1 , X_2) = 0
			\]
		\end{theorem}

		Also folgt 

		\begin{align*}
			Var(X_1 + X_2) &= Var(X_1) + Var(X_2) + 2 \cdot cov(X_1 ,X_2)\\
							&= Var(X_1) + Var(X_2)
		\end{align*}


		\subsubsection*{Wie werden Erwartungswerte aufsummiert?}
			\[
				E(a \cdot X_1 + b \cdot X_2) = a \cdot \textbf{E} X_1 + b \cdot \texbf{E}X_2
			\]

	\subsection*{a) $X_i$ auf (0,1) geleichverteilt sind}
		\subsubsection*{Wie wird Gleichverteilung definiert?}
			\begin{definition}{Gleichverteilung}
				$X \~ R(a,b)$.X hat die Dichtefunktion:

				\[
					\begin{cases} 
						0 &, falls x<a\\
						\frac{1}{b-a} &, falls a \leq  x < b\\
						0 &, falls x \geq b

				\]
			\end{definition}

			
		\subsubsection*{Wie wird ihr Erwartungswert definiert?}
			\begin{definition}{Erwartungswert $X$ stetig}
				\textbf{E}X = \int_{-\infty}^{\infty}x \dot f(x) dx
			\end{definition}
		Also :
			\begin{align*}
				EX &= \frac{1}{b-a}\int_{a}^{b}xdx \\ &= \frac{b^2 - a^2}{2(b-a)}\\ &= \frac_{a+b}{2}
			\end{align*}



\end{document}