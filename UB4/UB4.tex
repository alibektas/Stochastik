\documentclass{article}

\usepackage{amsmath}


\setlength{\parindent}{15pt}

\def\mathify#1{\ifmmode{#1}\else\mbox{$#1$}\fi} % guarantee math mode
\newcommand{\ceil}[1]{\mathify{\left\lceil {#1}\right\rceil}}
\newcommand{\floor}[1]{\mathify{\left\lfloor {#1}\right\rfloor}}

\title{Übungsblatt 4}
\author{Richardo Adrian Budianto  583669 \and Junhyuk Ko  531806 \and Ali Bektas 588063 }

\begin{document}
\maketitle



\section*{Aufgabe 12}
\subsection*{a)}
\[ P(B_i | A_k) = \binom{k}{i} p^i (1-p)^{k-i}  \]
\subsection*{b)}
\[  P(B_i | A_k) \cdot P(A_k) =  \binom{k}{i} p^i (1-p)^{k-i} \frac{\lambda^k}{k!}*e^{-\lambda}  \]
\subsection*{c)}
\[ \sum_{k=1}^{\infty} \binom{k}{i} p^i (1-p)^{k-i} \frac{\lambda^k}{k!}*e^{-\lambda}  \]

\section*{Aufgabe 13}
\begin{align*}
(i \geq 1) &= \binom{n}{i} \frac{n}{N}^i (1-\frac{n}{N})^{n-1}\\
1 - (i=0) &= 1 - \binom{n}{0}(\frac{n}{N}^0)(1-\frac{n}{N})^n\\
 	&= 1 - (1 - \frac{n}{N})^n\\
\end{align*}

\section*{Aufgabe 14}
\subsection*{Idee:}
Es sind mindestens zwei Würfe notwendig um davon zu reden , was die Wahrscheinlichkeit für den Gewinn von Spieler A in der kommenden Runde. Die Wahrscheinlichkeit dafür ist davon abhängig , dass das Spiel bisher unentschieden ist. Im Antritt von zwei Spielern kommen verschiedene Muster vor , von denen die Unentschiedenheit abhängig ist. Wir untersuchen zunächst das Verhalten von unentschiedenen Spielen. 

Sei $1:= z $ und $0:= w $.\\
Sei die erste Runde die Runde , in der die Münze zum dritten Mal geworfen ist.\\
Sei $x \subset \{0,1\}^{*}$ die Binärstring der Münzwürfe.\\

Definiere $P(x \neq A , x\neq B | x = y01 , x = y00) := $ Die Wkt ,  x ist nicht die Binärstring die vom Spieler A gewählt ist und x ist nicht die vom B gegeben x endet mit 01 oder 00. 

	
\subsection*{a)}
zzw gegen wzz $\implies$ 110 gegen 011

\[
P(x \neq 110 , x \neq 011 | x = y11) = \frac{\{\textbf{11}1\}}{2} = \frac{1}{2} 
\]
\[
P(110  gewinnt) = \frac{1}{8} \cdot \sum_{i=0}^{\infty} \frac{1}{2}^i = \frac{1}{4} \\ 
\]

\subsection*{b)}
011 gegen 001

\begin{align}
P(x \neq 011 , x \neq 001 | x = y01) &= \frac{\{\textbf{10}1\}}{2} = \frac{1}{2} \\
P(x \neq 011 , x \neq 001 | x = y10) &= \frac{\{\textbf{11}0,\textbf{01}0\}}{2} = 1\\
P(x \neq 011 , x \neq 001 | x = y11 , x = y01) &= \frac{\{\textbf{11}1,\textbf{10}1\}}{4} = \frac{1}{2}\\
P(x \neq 011 , x \neq 001 | x = y11 , x = y10) &= \frac{\{\textbf{11}1,\textbf{11}0,\textbf{01}0\}}{4} = \frac{3}{4}
\end{align}

Hier hören wir auf da die Präfixe 11 und 01 von (4) uns auf (3) zurückführen. 

\begin{align}
P(Gewinn) &= \frac{1}{8} + \frac{1}{8} \frac{1}{2} + \frac{1}{8}\frac{1}{2}\cdot 1 + \frac{1}{8}\frac{1}{2}\cdot 1 \cdot \frac{1}{2} + \frac{1}{8}\frac{1}{2}\cdot 1 \cdot \frac{1}{2} \frac{3}{4} \\
&= \dots +  \frac{1}{8}\frac{1}{2}\cdot 1 \cdot \frac{1}{2} \frac{3}{4} \dots \frac{1}{2} \frac{3}{4} \cdot \frac{1}{2} \frac{3}{4} \dots \\
&= \sum_{i=1}^{\infty} \frac{1}{8}_{i-1_+} \cdot \frac{1}{2}_{i-2_+} \cdot 1_{i-3_+} \cdot \frac{1}{2}^{\floor{\frac{i-3}{2}}} \cdot \frac{3}{4}^{\ceil{\frac{i-3}{2}}}
\end{align}



\subsection*{c)}
wwz gegen zww $\implies$ 001 gegen 100


\begin{align*}
P(x \neq 001 , x \neq 100 | x = y00) &= \frac{ \{ \textbf{00}0 \} }{ 2 } = \frac{1}{2} 
\end{align*}

\[ P(Gewinn) =  \frac{1}{8}  \sum_{i=0}^{\infty} \frac{1}{2} = 0.25 \]

\subsection*{d)}
zww gegen zzw $\implies$ 100 gegen 110


\begin{align*}
P(x \neq 100 , x \neq 110 | x = y10) &= \frac{ \{ \textbf{01}0 \} }{ 2 } = \frac{1}{2} \\
P(x \neq 100 , x \neq 110 | x = y01) &= \frac{ \{ \textbf{10}1 , \textbf{00}1 \} }{ 2 } = 1\\
P(x \neq 100 , x \neq 110 | x = y10 , x = y00) &= \frac{ \{ \textbf{01}0 , \textbf{00}0 \} }{ 4 } = \frac{1}{2}\\
P(x \neq 100 , x \neq 110 | x = y01 , x = y00) &= \frac{ \{ \textbf{00}1 , \textbf{10}1 , \textbf{00}0 \} }{ 4 } = \frac{3}{4}
\end{align*}

P(Gewinn) ist derselbe wie in (a).



\end{document}  