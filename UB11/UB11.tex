\documentclass{article}

\usepackage{amsmath}
\usepackage{amssymb}

\title{Übungsblatt 11}
\author{Richardo Adrian Budianto  583669 \and Junhyuk Ko  531806 \and Ali Bektas 588063 }


\begin{document}
	
	\maketitle
	\section*{Aufgabe 33}
		\subsection*{i)}
			\begin{align*}
				&1-\Phi(\frac{y-500}{\sqrt{250}}) < 0.01\\
				&-\Phi(\frac{y-500}{\sqrt{250}}) < -0.99\\
				&\Phi(\frac{y-500}{\sqrt{250}}) > 0.99\\
				&\Rightarrow \frac{y-500}{\sqrt{250}} > 2.33\\
				&y>536
			\end{align*}
		\subsection*{ii)}
			\begin{align*}
				&1-\Phi(\frac{510-500}{\sqrt{1000*0.25}})\\
				&1-\Phi(0.63)\\
				&1-0.735\\
				&0.265
			\end{align*}
		\subsection*{iii)}
			Sei X der erste Zug und Y der zweite.
			\begin{align*}
				&P(X>536 , Y>550)=\\
				&P(X>536 \lor X \leq 450)=\\
				&=1-\Phi(\frac{536-500}{\sqrt{250}})+\Phi(\frac{450-500}{\sqrt{(250)}})\\
				&=1-0.9885-0.00079\\
				&=0.01071
			\end{align*}
	\section*{Aufgabe 34}
		\subsection*{i)}
			\begin{align*}
				&n \geq (\frac{\sqrt{6.25}\Phi^{-1}(1-0.1)}{0.1})^2\\
				&n \geq (\frac{(2.5)(1.28)}{0.1})^2\\
				&n \geq 1024
			\end{align*}
		\subsection*{ii)}
			\begin{align*}
				&P(|Y-EY|\geq \epsilon ) \leq \frac{\text{Var }Y}{\epsilon^2}\\
				&\rightarrow 1-P(|Y-EY|\geq \epsilon) \geq 1 - \frac{\text{Var }Y}{\epsilon^2}\\
				&\rightarrow 1-P(|Y-EY|\geq \epsilon) \geq 1 - \frac{6.25}{n*\epsilon^2} \geq 0.9\\
				&n\geq 6250
			\end{align*}

	\section*{Aufgabe 35}
		\[
			\prod^n_{i=1} f_{X_i}(x_i) = 
					\begin{cases} 
						\frac{1}{(b-a)^n} ,&falls 0\leq x_i \leq b \forall x_i\\
						0 ,&sonst.
					\end{cases}
		\]

		Setze $\theta := b-a $. Dann gilt für ML-Schätzungen 

		\[
			a  : min(x_1 , \dots , x_n)\\
			b  : max(x_1 , \dots , x_n)
		\] 
\end{document}